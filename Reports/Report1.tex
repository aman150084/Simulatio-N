\documentclass{article}
\usepackage{verbatim}
\title{Daily Progress Report}
\author{Arkavo Hait, Sagnik Bhattacharya, Aman Deep Singh}
\begin{document}
\maketitle
%\tableofcontents
%\section*{What we have done}
\section*{Summary}
Prepared a rough code which would be used in the final program.
The Code only applies to one force: gravity
However it is still text based but prints the results (position and velocity data) into a text file. It is still a sequential program, but in a form that would be easy to carry over into parallel form.
\section*{Detailed Report}
\subsection*{May 16, 2016}
Created repo on GitHub
\subsection*{May 17, 2016}
Created abstract idea for main code
\subsection*{May 25, 2016}
\subsubsection*{Arkavo Hait}
Started learning CUDA, from documentation and a tutorial (link in repo).
\subsubsection*{Sagnik Bhattacharya}
Started a course on Udacity on Parallel programming using CUDA. Completed Lesson 1: The GPU programming model.
\subsubsection*{Aman Deep Singh}
Started watching a Video Tutorial series(by David Gohara) focused on OpenCL. Completed 3 Episodes : `Introduction to OpenCL', `OpenCL Fundamentals' and `Building an OpenCL Project'.
\subsection*{May 26, 2016}
\subsubsection*{Arkavo Hait}
Continued above tasks. Tried out a simple CUDA program.
\subsubsection*{Sagnik Bhattacharya}
Continued with above Udacity course. Reached Lesson 2: GPU Hardware and Parallel Communication Patterns.
\subsubsection*{Aman Deep Singh}
Continued with the Video Tutorial Series. Moved on to Episode 4:`Memory Layout and Access'
\subsection*{May 27, 2016}
\subsubsection*{Arkavo Hait}
Created skeleton code for bodies.
\subsubsection*{Sagnik Bhattacharya}
Made skeleton code Gravity2.c for taking input of planet data.
\subsubsection*{Aman Deep Singh}
Started following an OpenGL tutorial. Tried to make an OpenGL project using CMake but couldn't do it on Ubuntu due to a lot of Errors. Started trying it on windows. There were many more errors.
\subsection*{May 28, 2016}
\subsubsection*{Arkavo Hait}
\begin{enumerate}
	\item Made a concept skeleton CUDA code for planetary movement.
	\item Added code functionality to enable output of data to file (List.txt)
	\item Began converting seq.c to sequence.h
\end{enumerate}
\subsubsection*{Sagnik Bhattacharya}
\begin{enumerate}
	\item Began making a sequential program for N-body simulation, so that that functions created for that program could be used in the CUDA program that will be created.
	\item Removed errors in above program. Debugged a segfault, added code to calculate the acceleration of a body given the the positions of the rest of the bodies, and to calculate the position and velocity of each planet after each iteration.
\end{enumerate}
\subsubsection*{Aman Deep Singh}
\begin{enumerate}
    \item Completed the Video Tutorial Series on OpenCL.
    \item Tried to make an OpenGL project using Visual Studio Express 2015 and CMake and faced errors related to missing Header Files.
\end{enumerate}
\subsection*{May 29, 2016}
\subsubsection*{Arkavo Hait}
\begin{enumerate}
	\item Began to make a library to facilitate creation of CUDA code.
	\item Made reports.
\end{enumerate}
\subsubsection*{Sagnik Bhattacharya}
Made reports. Did some debugging of skeleton code.
\begin{enumerate}
	\item Removed error in function addVec, that did not return the vector formed by addition of two given vectors.
	\item Removed error in seq.c that stopped the time from updating successfully between iterations.
	\item Added code for time stamp to appear with each update of the planetArray.
\end{enumerate}
\subsubsection*{Aman Deep Singh}
\begin{enumerate}
    \item Learned how to make header files and made a sample header file for basic functions like  Addition, Subtraction, Multiplication and Division.
    \item Made Reports.
\end{enumerate}

\begin{comment}
\section{What we will do}
Convert the C code to CUDA code to be essential in Parallel computing
Create a GUI to make things visual.
Add other forces and their effects in the future.
We are still workin on the basics of CUDA programming and will covert it to CUDA language soon.
Converting the initial code to a function library to ease up the process and debugging
\end{comment}
\end{document}
